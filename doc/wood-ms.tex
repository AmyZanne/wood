\documentclass[12pt]{article}
\usepackage{ms}
\usepackage{natbib}
\usepackage{lineno}
\modulolinenumbers[5]
\linenumbers

\title{How much of the world is woody / The world is woodier than you
  think}
\author{NESCent Tempo and Mode Working Group}
\date{}
\affiliation{\noindent
Fill this in here\ldots}
\runninghead{Wood wood wood?}

\begin{document}
\mstitlepage
\parindent=1.5em
\addtolength{\parskip}{.3em}

\section{Introduction}

Some species produce aboveground stems that persist for multiple years, others do not, and this represents perhaps the most profound contrast among terrestrial plants and ecosystems: the difference between a forest and a grassland is the presence or absence of wood. Classifying plants as woody or herbaceous dates back at least to Enquiry into Plants by Theophrastus of Eresus (371 - 287 BC) a student of Plato and Aristolte, who began his investigation into plant form and function by classifying the hundreds of plants in his garden into woody and herbaceous categories \citep{theophrastus1916enquiry}.  We now know a bit more about wood than Theophrastus: evolutionarily wood dates to the early Devonian \citep[~400 mya;][]{gerrienne2011simple}, and the evolution of woodiness across the phylogenetic tree depends on climate and differs among the major plant clades (Zanne/ Beaulieu et al.), but basic questions still remain.

Now in the age of global bioinformatics, building a global picture of biodiversity has become possible.  One important question that attracted a great deal of interest is the quest to estimate the number of species on the planet \citep{erwin1991many, may1988many, stork1993many}.  These statistical estimates have gotten better with time, narrowing in most recently on a number around 8.7 million species with ~400,000 of those plants (refs).  It is also becoming clear that in a world with 8.7 million species, the clear next step toward a systematic understanding of that diversity is a system to understand the structure of that diversity.  Two complementary approaches have been proposed: phylogenetic reconstruction of shared descent \citep{smith2011understanding} and a functional approach following Theophrastus (more recently in the tradition of \citep{grime1979plant, weiher2009challenging, westoby2002plant}): essentially asking how many different ways do species make their living and how many species are in each category?

Global taxonomic and phylogenetic efforts, although still not perfect, are much closer to a global completeness compared to the functional approach.   There has been a great deal of progress is assembling both taxonomies and phylogenies for the huge numbers of species including plants \citep{smith2011understanding} but despite the importance of these functional databases for ecosystem services and conservation, we are only just beginning to assemble analogous databases that describe the functional characteristics of those species \citep{kattge2011try}.   The fact that phylogenetic efforts are a good deal more advanced than functional one is surprising given that the 2000-year history of the functional approach and the 250-year history of taxonomy.  Interestingly, because taxonomic efforts are ahead of functional efforts, they can be used in combination with functional database to formulate estimates of the global structure of functional biodiversity.

Now in the age of global bio-informatics, we were interested in simply re-asking Theophrastus's 2000 year old question at the global scale: how many of the world's plant species are woody?  

\section{Methods}

\section{Results}

\section{Discussion}

\bibliographystyle{ecology}
\bibliography{wood.bib}

\end{document}

%%% Local Variables:
%%% mode: latex
%%% TeX-master: t
%%% TeX-PDF-mode: t
%%% End:
