\documentclass[12pt]{article}
\usepackage{ms}
\usepackage{natbib}
\usepackage{lineno}
\modulolinenumbers[5]
\linenumbers

\title{How much of the world is woody?}
\author{
Richard G. FitzJohn$^{*,1}$\\ Matthew W. Pennell$^{*,2,3}$,\\ Amy E. Zanne$^{4}$,\\ William K. Cornwell$^{5}$
}
\date{}
\affiliation{\noindent
$^*$ These authors contributed equally}
\runninghead{The proportion of woody species}

\begin{document}
\mstitlepage
\parindent=1.5em
\addtolength{\parskip}{.3em}

\section{Introduction}

Some species produce aboveground stems that persist for multiple years, others do not, and this represents perhaps the most profound contrast among terrestrial plants and ecosystems: the difference between a forest and a grassland is the presence or absence of wood. Classifying plants as woody or herbaceous dates back at least to \textit{Enquiry into Plants} by Theophrastus of Eresus (371 - 287 BC) a student of Plato and Aristolte, who began his investigation into plant form and function by classifying the hundreds of plants in his garden into woody and herbaceous categories.  We now know a bit more about wood than Theophrastus: evolutionarily wood dates to the early Devonian \citep[~400 mya;][]{gerrienne2011simple}, and the evolution of woodiness across the phylogenetic tree depends on climate and differs among the major plant clades (Zanne/ Beaulieu et al.), but basic questions still remain.

Now in the age of global bioinformatics, building a global picture of biodiversity has become possible.  One important question that attracted a great deal of interest is the quest to estimate the number of species on the planet \citep{may1988many,erwin1991many, stork1993many, mora2011plos}.  These statistical estimates have gotten better with time, narrowing in most recently on a number around 8.7 million species with around 300,000 of those plants \citep{mora2011plos}.  It is also becoming clear that in a world with 8.7 million species, the clear next step toward a systematic understanding of that diversity is a system to understand the structure of that diversity.  Two complementary approaches have been proposed: phylogenetic reconstruction of shared descent \citep{smith2011understanding} and a functional approach following Theophrastus \citep[more recently in the tradition of][]{grime1979plant, weiher2009challenging, westoby2002plant}: essentially asking how many different ways do species make their living and how many species are in each category?

Global taxonomic and phylogenetic efforts, although still not perfect, are much closer to a global completeness compared to the functional approach.   There has been a great deal of progress is assembling both taxonomies and phylogenies for the huge numbers of species including plants \citep[e.g.][]{smith2011understanding} but despite the importance of these functional databases for ecosystem services and conservation, we are only just beginning to assemble analogous databases that describe the functional characteristics of those species \citep{Kattge2011TRY}.   The fact that phylogenetic efforts are a good deal more advanced than functional one is surprising given that the 2000-year history of the functional approach and the 250-year history of taxonomy.  Interestingly, because taxonomic efforts are ahead of functional efforts, they can be used in combination with functional database to formulate estimates of the global structure of functional biodiversity.

Now in the age of global bio--informatics, we were interested in simply re-asking Theophrastus' 2000 year old question at the global scale: how many of the world's plant species are woody? At the outset, we assumed that the answer to such a rudimentary question would likely be generally known. But upon consulting the literature, we discovered that surprisingly little had been written on this topic. We then posed the question to several taxonomic experts and received a tremendous variety of answers in response. Here we undertake the first attempt, to our knowledge, to try and quantify the proportion of woodiness across the world's flora. We also took the unconventional approach of coupling our analysis of the data with a survey in which we asked our question to the broader community of botanists and other biologists. The motivation for this was to determine whether a consensus answer did exist---even if not formalized in the literature---and if so, whether it was consistent with the data.

Though the experimental design of our survey was not entirely ideal (e.g. we relied on voluntary response) and certainly biased in terms of who responsed, we were none-the-less able to demonstrate the novelty of this project. Not only was there a huge variance in responses, suggesting that a consensus did not exist, but the mean and median estimates from the survey were far from our those estimate obtained from analyzing available data. This evidence suggests that researchers intuitions regarding the proportion of woody plants may be severly misinformed.

\section{Methods}
\subsection{Survey}
In order to determine if a consensus existed and whether estimates were correlated with general familiarity with plants, professional training or geographic region, we created an English-language survey consisting of the following questions:

\begin{enumerate}

\item What percentage of the world's vascular plant species are woody?

\item How would you rate your familiarity with plants?

	\begin{itemize}
	
		\item Very Familiar
	
		\item Familiar
	
		\item Somewhat Familiar
	
		\item What's a Plant?
	
	\end{itemize}
	
\item How much formal training have you received in botany?

	\begin{itemize}
	
		\item Postgraduate degree in botany or a related field
		
		\item Partially complete postgraduate degree in botany or a related field
		
		\item Undergraduate degree in botany or a related field
		
		\item Some botany courses at either an undergraduate or postgraduate level
		
		\item No formal training in botany
		
	\end{itemize}
	
\item In what country did you receive your biology/botany training?

\end{enumerate}

This survey was created using google docs (original survey included as Supplementary Material) and distributed to the community via several electronic mailing lists with wide circulation among biologists. These included \emph{EvolDir}, \emph{ECOLOG}, \emph{r-sig-phylo}, \emph{Taxacom}, \emph{Herbaria}, Local Departmental Mailing List (e.g. Institute for Bioinformatics and Evolutionary Studies, University of Idaho). We also leveraged social-networking platforms such as \textsc{google+}, \textsc{twitter} and \textsc{facebook} to reach as broad an English-language audience as we could. In order to increase representation of survey responses from Latin American, we translated the survey into Portugese (thanks to Rafael Maia) and distributed to various Brazilian biology \textsc{facebook} groups and university mailing lists.

Details on survey analysis here [Will]


\section{Results}

\subsection{Estimation}

\subsection{Survey}

We received 293 responses from 31 countries. The mean estimate for the proportion of woodiness was 31.7\% and a median estimate of 30\%. The standard deviation of the estimates was 17.7. The respondent's degree of training in botany had no effect on their estimate (Fig XX). There was a significant effect ($p<0.01$) of the respondent's  familiarity with plants on the estimates they gave though this effect seemed to be driven primarily by respondent's with very little familiarity with plants (the ``What's a Plant?'' category), whose estimates tended to be the lower (less woody) than the respondent's as a whole.

\section{Discussion}

A global picture of biodiversity requires taxonomic, genetic and functional components.  In many ways research into function is the oldest of these lines of research, but today our functional understanding of global biodiversity is not as developed as our taxonomic and genetic understanding. Woodiness is a fundamental plant trait for many reasons: the presence or absence of wood  structures the main difference between terrestrial biomes; woody and herbaceous species also evolve at different rates \citep{SmithDonoghue};  there are also important evolutionary interactions with climate (Zanne); and wood itself has crucial effects on the global carbon cycle \citep{Cornwellwood}.  The question this paper asks---how many of the world's plant species are woody---is an extremely basic global description of global functional diversity.

[DISCUSS RESULTS OF ANALYSIS]

We acknowledge that conducting a poll of the research community is certainly unconventional for a scientific research paper; the a priori predictions of other researchers are usually irrelevant. However, as the question we sought to answer was so basic, we were interested in whether a general consensus existed as to the answer, even if it did not appear in the peer-reviewed literature. Somewhat to our surprise, not only did a broad consensus not exist (there was a large variance in the estimates) but researchers systematically underestimated the proportion of the world's flora that is woody. This perceptual bias was present in researchers with all manner of formal education and familiarity with plants, with the lone excpetion being that people very unfamililar with plants tended to think of the world as even less woody than average. We have a number of admittedly speculative hypotheses as to why this may the case. One is that researchers may have a temperate bias; we know that the proportion of woody plants is greater in the tropics than in the temperate regions \citep{Molesheihgt}. However, in our informal survey, respondents from tropical countries were no more accurate in their predictions than their temperate colleagues (results not shown). Another possible cause for our herb-centric view of the world is that many common cultivated plants are herbaceous---this is true wherever we are in the world. We are much more likely to spend time in the garden than the rainforest. 

Whatever the reason for the discordance between the survey results and our data analysis, the fact that such a discordance exists further supports our claim that this most basic of plant natural history facts is not generally known.   



\bibliographystyle{ecology}
\bibliography{wood.bib}

\end{document}

%%% Local Variables:
%%% mode: latex
%%% TeX-master: t
%%% TeX-PDF-mode: t
%%% End:
