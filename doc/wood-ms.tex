\documentclass[12pt]{article}
\usepackage{ms}
\usepackage{natbib}
\usepackage{lineno}
\usepackage{graphicx}
\modulolinenumbers[5]
\linenumbers

\title{How much of the world is woody?}
\author{
Richard G. FitzJohn$^{*,1}$\\ Matthew W. Pennell$^{*,2,3}$,\\ Amy E. Zanne$^{4}$,\\ William K. Cornwell$^{5}$
}
\date{}
\affiliation{\noindent
$^*$ These authors contributed equally}
\runninghead{The proportion of woody species}

\begin{document}
\mstitlepage
\parindent=1.5em
\addtolength{\parskip}{.3em}

\begin{abstract}
The woody/herbaceous divide is perhaps the most fundamental axis of functional diversity in plants. The importance of variation in this trait was recognized over two thousand years ago by the Greek philosopher Theophrastus who attempted to classify the plants in his garden as either woody or herbaceous. Though our knowledge of the taxonomic structure of global diversity has increased tremendously since the era of Theophrastus, some very rudimentary questions regarding functional diversity remain. Here we sought to address perhaps the most basic of all of these questions: how many of the world's species are woody? Surprisingly this question has never been comprehensively addressed. Here we address this question by combining the largest plant functional trait data base to date with knowledge of the taxonomic distribution of diversity to account for the many species for which we do not have data. Using various sampling procedures, we estimate the proportion ``woodiness'' among the world's vascular plants to be between 46\% and 48\%. We were also curious as to whether a consensus answer to our question existed among botanists---even if not formally documented in the peer-reviewed literature---and if so, whether it was consistent with the data. To investigate this we took the unconventional approach of coupling our analysis of the data with a survey in which we posed our question to the broader community of botanists and other biologists.  We found that not only did no such consensus exist, but that researchers tended to systematically underestimate the proportion of the world's flora that was woody; that true was regardless of the respondents' training or familiarity with plants. The results of our survey highlight just how global datasets can show patterns at odds with convensional wisdom.  Global functional diversity is much woodier than we thought.  
\end{abstract}

\section{Introduction}

Some species produce aboveground stems that persist for multiple years, others do not, and this represents perhaps the most profound contrast among terrestrial plants and ecosystems: the difference between a forest and a grassland is the presence or absence of wood. The recognition of the fundamental importance of this divide dates back at least to \textit{Enquiry into Plants} by Theophrastus of Eresus (371 - 287 BC) a student of Plato and Aristolte, who began his investigation into plant form and function by classifying the hundreds of plants in his garden into woody and herbaceous categories \citep{theophrastus1916enquiry}.  We now know a bit more about wood than Theophrastus: wood is thought to have originated in the early Devonian \citep[~400 mya;][]{gerrienne2011simple}, and the evolution of woodiness across the phylogenetic tree depends on climate and differs among the major plant clades (Zanne/ Beaulieu et al.), but basic questions still remain. [NOTE: We should fix this sentence as well as add a more thorough description of what exactly wood is]

[NEED TRANSITION] Now in the age of global bioinformatics, building a global picture of biodiversity has become possible.  One important question that attracted a great deal of interest is the quest to estimate the number of species on the planet \citep{may1988many,erwin1991many, stork1993many, mora2011plos}.  These statistical estimates have gotten better with time, narrowing in most recently on a number around 8.7 million species with around 300,000 of those plants \citep{mora2011plos}.  It is also becoming clear that in a world with 8.7 million species, the clear next step toward a systematic understanding of that diversity is a system to understand the structure of that diversity. There are two complementary approaches to address this problem: a phylogenetic approach, in which diversity is structured according to shared ancestry and a functional approach, following Theophrastus \citep[more recently in the tradition of][]{grime1979plant, weiher2009challenging, westoby2002plant}, that essentially asks how many different ways do species make their living and how many species are in each category?

Global taxonomic and phylogenetic efforts, although still not perfect, are much closer to being globally complete compared to the functional approach.   There has been a great deal of progress is assembling both taxonomies and phylogenies for the huge numbers of species including plants \citep[e.g.][]{smith2011understanding}. But despite their importance for understanding ecosystem services and making conservation decisions, we are only just beginning to assemble analogous databases that describe the functional characteristics of those species \citep{Kattge2011TRY}. The fact that phylogenetic efforts are a good deal more advanced than functional ones is surprising given the 2000-year history of the functional approach and the 250-year history of the taxonomic approach.  

%Interestingly, because taxonomic efforts are ahead of functional efforts, they can be used in combination with functional database to formulate estimates of the global structure of functional biodiversity.

Now in the age of global bio-informatics, we were interested in simply re-asking Theophrastus' 2000 year old question at the global scale: how many of the world's plant species are woody? At the outset, we assumed that the answer to such a rudimentary question would likely be generally known. But upon consulting the literature, we discovered that surprisingly little had been written on this topic. We then posed the question to several taxonomic experts and received a tremendous variety of answers in response suggesting that this fundamental aspect of plant functional diversity was surprisingly poorly understood. Here we undertake the first attempt, to our knowledge, to try and quantify the proportion of woodiness across the world's flora. We also took the unconventional approach of coupling our analysis of the data with a survey in which we asked our question to the broader community of botanists and other biologists. The motivation for this was to determine whether a consensus answer did exist---even if not formalized in the literature---and if so, whether it was consistent with the data.

Though the experimental design of our survey was not ideal (e.g. we relied on voluntary response) and certainly biased in terms of who responsed, we were none-the-less able to demonstrate the novelty of this project. Not only was there a huge variance in responses but the mean and median estimates from the survey were far from our those estimate obtained from analyzing available data. This evidence suggests that researchers' intuitions regarding the proportion of woody plants may be severly misinformed. [THE END OF THE INTRODUCTION IS WEAK]

\section{Methods and Materials}

\subsection{Dataset}

We used a recently assembled database with growth--form data for 48062 species, which is the largest database assembled to date.  In this dataset woodiness is defined as the persistence of a perennial above--ground stem, which includes palms and tree ferns as woody---species that are excluded by alternate definitions of ``wood''.  There are a number of species which are intermediate in form Beaulieu et al. (in revision); in this dataset 633 species (1.3\%) were coded as variable.  For the formal analysis we excluded these species.  Because the effort to organize plant taxonomy, especially synonymy, is on-going, there was uncertainty regarding the status of many of these plant names.  We matched this database with the accepted names from the plant list leading to 36664 records with documented taxonomy---14833 herbs and 21831woody species.  

%WKC:  compare family sampling to total number of families? Somehow show the dataset is comprehensive?

\subsection{Estimating the fraction of species that are woody}

% RGF: I've put the entire results part here.  I think that the paper
% is going to be too short to justify a methods/results split,
% probably.

[Describe the database collection, and give a few summary statistics
about the database here]

We cannot simply use the fraction of species within our trait
data set (21,743 / 36,238 = 60\%) as these records may represent a
biased sample of vascular plants.
% sum(dat.g$W) # woody
% sum(dat.g$K) # known
% sum(dat.g$W) / sum(dat.g$K) # fraction
For example, most Orchidaceae are probably herbaceous; we have only
two records of woodiness among the 1,574 species for which we have
data.
% i.orc <- dat.g$family == "Orchidaceae"
% sum(dat.g$K[i.orc]) # known Orchidaceae
% sum(dat.g$W[i.orc]) # woody Orchidaceae
However, the fraction of Orcidaceae with known data (1,574 / 27,104 =
6\%)
% sum(dat.g$K[i.orc]) # known Orchidaceae
% sum(dat.g$N[i.orc]) # number of Orchidaceae
% sum(dat.g$K[i.orc]) / sum(dat.g$N[i.orc]) # knowledge rate of Orchidaceae
is much lower than the overall rate of knowledge over all vascular
plants (36,238 / 274,141 = 13\%), which will downwardly bias the
global estimate of woodiness.
% sum(dat.g$K) # known vascular
% sum(dat.g$N) # total vascular
% sum(dat.g$K) / sum(dat.g$N) # knowlege rate for all vascular plants
%
Conversely, systematic undersampling of tropical species, believed to
be more woody than temperate species \citep{Molesheihgt}, will bias
the global woodiness estimate downwards.

Most genera are either all-woody or all-hebacious.  Among the 748
genera with at least 10 records, 392 are entirely woody, 248 are
entirely herbaceous, and only 50 have between 10\% and 90\% woody
species.  As a result, knowing the state of a handful of species
within a genus can give a reasonable guess at the remaining species.
% tmp <- dat.g$p[dat.g$K >= 10] # genera with 10 records
% sum(tmp == 1) # 100% woody
% sum(tmp == 0) # 100% herbaceous

To impute the state of the species with unknown states, we used a
simple procedure where we assumed that known species were sampled
without replacement a pool of species, and then sampled the
composition of this pool.  For genera with no known
states, we sampled a woodiness fraction from the empirical
distribution of woodiness fractions for other genera within a family. This procedure is described in detail in the appendix to this paper.

% RGF: I'm going to start with the results tomorrow.
\begin{figure}[p]
  \centering
  \includegraphics{fraction-by-genus}
  \caption{Distribution of the fraction of woodiness among genera.
    The distribution of the fraction of species that are woody within
    a genus (among genera) is strongly bimodal (panel a -- showing
    genera with at least 10 species only).
    % 
    The two different sampling approaches generate distributions that
    differ in their bimodality (panel b).  When assuming species are
    sampled with replacement from some pool, but having no prior on
    the fraction of woodiness within the pool generates a broad
    distribution with many polymorphic genera (blue bars), while
    sampling with replacement assuming that species are drawn from a
    pool of species that has a fraction of woody species equal to the
    observed fraction of woodiness generates a strongly bimodal
    distribution.}
  \label{fig:distribution-genera}
\end{figure}



\subsection{Survey}
In order to determine if a consensus existed and whether estimates were correlated with general familiarity with plants, professional training or geographic region, we created an English-language survey consisting of the following questions:

\begin{enumerate}

\item What percentage of the world's vascular plant species are woody?

\item How would you rate your familiarity with plants?

	\begin{itemize}
	
		\item Very Familiar
	
		\item Familiar
	
		\item Somewhat Familiar
	
		\item What's a Plant?
	
	\end{itemize}
	
\item How much formal training have you received in botany?

	\begin{itemize}
	
		\item Postgraduate degree in botany or a related field
		
		\item Partially complete postgraduate degree in botany or a related field
		
		\item Undergraduate degree in botany or a related field
		
		\item Some botany courses at either an undergraduate or postgraduate level
		
		\item No formal training in botany
		
	\end{itemize}
	
\item In what country did you receive your biology/botany training?

\end{enumerate}

This survey was created using google docs (original survey included as Supplementary Material) and distributed to the community via several electronic mailing lists with wide circulation among biologists. These included \emph{EvolDir}, \emph{ECOLOG}, \emph{r-sig-phylo}, \emph{Taxacom}, \emph{Herbaria}, and local departmental mailing lists (e.g. Institute for Bioinformatics and Evolutionary Studies, University of Idaho). We also leveraged social-networking platforms such as \textsc{google+}, \textsc{twitter} and \textsc{facebook} to reach as broad an English-language audience as we could. In order to increase representation of survey responses from Latin American, we translated the survey into Portugese (thanks to Rafael Maia) and distributed it to various Brazilian biology \textsc{facebook} groups and university mailing lists. (The survey we sent out is included as Online Supplementary Material.)

[Details on survey analysis here]


\section{Results and Discussion}

A global picture of biodiversity requires taxonomic, genetic and functional components.  In many ways research into function is the oldest of these lines of research, but today our functional understanding of global biodiversity is not as developed as our taxonomic and genetic understanding. Woodiness is a fundamental plant trait for many reasons: the presence or absence of wood  structures the main difference between terrestrial biomes; woody and herbaceous species also evolve at different rates \citep{SmithDonoghue};  there are also important evolutionary interactions with climate (Zanne); and wood itself has crucial effects on the global carbon cycle \citep{Cornwellwood}.  The question we ask in this paper---how many of the world's plant species are woody---is an extremely basic global description of global functional diversity.

[DISCUSS RESULTS OF ANALYSIS]

We acknowledge that conducting a poll of the research community is certainly unconventional for a scientific research paper; the a priori predictions of other researchers in the field are often considered irrelevant. However, as the question we sought to answer was so basic, we were interested in whether a general consensus existed as to the answer, even if it did not appear in the peer-reviewed literature. In response to our survey, we received 293 responses from 31 countries. The mean estimate for the proportion of woodiness was 31.7\% and a median estimate of 30\%. The standard deviation of the estimates was 17.7. The respondent's degree of training in botany had no effect on their estimate (Fig XX). There was a significant effect ($p<0.01$) of the respondent's  familiarity with plants on the estimates they gave though this effect seemed to be driven primarily by respondent's with very little familiarity with plants (the ``What's a Plant?'' category), whose estimates tended to be the lower (less woody) than the estimates from other categories. Taken as a whole, this is perhaps surprising. Not only did a broad consensus not exist (there was a large variance in the estimates) but researchers systematically underestimated the proportion of the world's flora that is woody. This perceptual bias was present in researchers with all manner of formal education and familiarity with plants. We have a number of admittedly speculative hypotheses as to why this may the case. One is that researchers may have a temperate bias; we know that the proportion of woody plants is greater in the tropics than in the temperate regions \citep{Molesheihgt}. However, in our informal survey, respondents from tropical countries were no more accurate in their predictions than their temperate colleagues (results not shown). Another possible cause for our herb-centric view of the world is that many common cultivated plants are herbaceous---this is true wherever we are in the world. We are much more likely to spend time in the garden than the rainforest. Whatever the reason for the discordance between the survey results and our data analysis, the fact that such a discordance exists further supports our claim that this most basic of plant natural history facts was not generally known.   

[CONCLUDING THOUGHTS]

\clearpage
\setcounter{secnumdepth}{1}
\appendix
\section{Sampling proceedure}

Suppose that for some genus there is a true number of woody species
$N_w$ among the $N$ species in the genus, and that we have randomly
sampled $n_w$ woody and $n_h$ species from this pool.
%
For example, with the genus \textit{Microcoelia} (Orchidaceae) all 13
species with known state are herbaceous ($n_w = 0$, $n_h = 13$, but we
we do not know the state of the remaining 17 species in the genus ($N
= 30$).  In general, we can't assume that these species are all
herbacious, even though odds are that most of them will be, so the
true number of woody species may lie between 0 and 17.

We used two different approaches for imputing the values of these
missing species.  First, we assumed that the ``known'' species were
sampled without replacement from a pool of species with $N_w$ woody
and $N_h$ herbaceous species ($N_w + N_h = N$).  The probability that
$x$ of the species woody ($x = 0, 1, \ldots, N
- n_w - n_h$) is proportional to
\begin{equation}
  \Pr(N_w = x) \propto {n_w + x \choose n_w}
  {N - (n_w + x) \choose n_h}
\end{equation}
For \textit{Microcoelia} this gives a 45\% probability that all
species are herbacious, and a 92\% chance that at most three species
are woody.

This approach probably overestimates the number of woody species in
this case, and in other cases (such as XXX where all YYY species with
known states are woody) will understimate the fraction of species that
are woody.  This corresponds to a very weak prior knowledge of the
shape of the distribution of the fraction of woody species within a
genus.  However, this distribution is strongly bimodal (FIGURE).  To
model the other extreme of sampling, we used an approch where we
computed the observed fraction of woody species ($p = n_w / (n_w +
n_h)$) and sampled the state of the unobserved species using a
binomial distribution, where the probability that $y$ of the $N - n_w
- n_h$ species with no known state are woody is
\begin{equation}
  \Pr(y = k) = {N - n_w - n_h \choose k} p^k (1-p)^{N - n_w - n_h - k}.
\end{equation}
In cases where all known species are woody (or herbaceous) this will
assign all unknown species to be woody (or herbaceous).  In
polymorphic cases this will give similar results method above.  This
approach corresponds to a very strong prior on the shape of the
distribution of woodiness among genera.
% RGF: This is really vague and I will think about it more when I'm
% less braindead.
While neither of these approaches is ``correct'', they probably
span the range of possible outcomes.

% RGF: This paragraph is the hardest to write for the sampling.  You
% may have to help here.
For genera where there was no information on woodiness for any
species, we sampled a fraction of species that might be woody from the
empirical distribution of woodiness fractions \textit{among genera}
within the same order.  We did this after imputing the missing species
values within those other genera.  So, if a genus is found in an order
with genera that had woodiness fractions of $[0, 0, .1, 1]$ we would
have approximately a 50\% chance of sampling a 0\% woodiness fraction
for a genus.  Given this woodiness fraction, we then sampled the
number of species that are woody from a binomial distribution with
this fraction as it's parameter.

We repeated the above sampling approach 1,000 times and report means
and confidence intervals over this distribution.

\bibliographystyle{ecology}
\bibliography{wood.bib}

\end{document}

%%% Local Variables:
%%% mode: latex
%%% TeX-master: t
%%% TeX-PDF-mode: t
%%% End:
