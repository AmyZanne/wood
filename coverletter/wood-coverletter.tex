\documentclass[a4paper,12pt]{article}

\setlength{\parskip}{1ex plus1pt} % threequarters
\RequirePackage[hmargin=3.5cm,vmargin=3cm]{geometry}
\usepackage{graphicx}
\pagestyle{empty}
\usepackage[osf]{mathpazo}

\begin{document}

{\raggedleft
  Matthew Pennell\\
  Institute for Bioinformatics and Evolutionary Studies\\
  University of Idaho\\
 Moscow, ID 83844, U.S.A.\\[2ex]
}

\vspace{3ex}

Dear Drs. Gibson and Sandhu,

We would like to submit our manuscript ``How much of the world is woody?'' for consideration as a ``Future Directions'' article in the Journal of Ecology. We sent a pre-submission inquiry via an email on June 29$^{th}$, 2013 and were encouraged to proceed with submitting to this journal.
					
The goal of our paper was to estimate the global percentage of woody species, a basic aspect of functional biodiversity that is (surprisingly) currently unknown. However, global trait datasets, such as the one we used for our analyses are taxonomically biased such that using the point estimate would be inaccurate. To address this problem we used a recently assembled database and a stochastic gap-filling approach. This allows us to estimate both the global proportion of species that are woody, along with uncertainty in the estimate. 

We are writing this as a ``Future Directions'' paper because the problem we have identified—the taxonomic bias of our data set --- is likely pervasive among large databases. Sampling bias can introduce very severe, but underappreciated and poorly-understood, biases in downstream analyses. The solution we develop in this paper (along with the accompanying R code that will be submitted with the paper) may be of broad utility for mitigating taxonomic sampling biases when investigating all sorts of macroecological and macroevolutionary questions using large trait databases.

As researchers are increasingly focusing on questions at a global scale, understanding biases in large datasets --- and finding ways to deal with them --- is in our view, an important area for future work. We hope you will consider our paper for publication and look forward to hearing from you.

\begin{flushright}
\vspace{2ex}
\hspace{.2\textwidth}Sincerely,\\
\hspace*{.3\textwidth}
Matthew Pennell
\end{flushright}

\end{document}
