\documentclass[a4paper,12pt]{article}

\setlength{\parskip}{1ex plus1pt} % threequarters
\RequirePackage[hmargin=3.5cm,vmargin=3cm]{geometry}
\usepackage{graphicx}

\usepackage{mathpazo}

\begin{document}

{\raggedleft
  Dr. Richard FitzJohn\\
  Department of Biological Sciences\\
  Macquarie University\\
  Sydney, NSW 2109, Australia\\[2ex]
  1 March 2013\\
}

\vspace{3ex}

Dear Prof XXX,

Please consider our manuscript ``How much of the world is woody?''
for publication.  This might be seen as an unconventional paper, but
on the other hand this is a very basic question in global functional
plant ecology.

As part of a working group at NESCent (the National Science Synthesis
Center in Durham, NC), we needed the global proportion of woody
species for an evolutionary analysis.  We were very surprised when we
were completely unable to find that information anywhere in the
literature.  We then surveyed our working group and a number of plant
experts who were at NESCent at that time, and the wide variance of
answers led us to conclude that there was really a research problem to
be solved.

To solve the problem we used a recently assembled database and a
stochastic gap--filling approach.  This allows us to estimate both the
global proportion of species that are woody, along with uncertainty in
the estimate.  Our answer is 47+/- 2, which is much higher than the
conventional wisdom.

In many ways we feel this paper is the functional analog to the series
of papers making statistical estimates on the number of species in the
world.  Like those papers, we feel that the simplicity of the question
posed in this paper may lead to wide interest in the result.

We hope that you will consider our work for publication.

\vspace{2ex}
\hspace{.2\textwidth}Yours sincerely,\\[2ex]
\hspace*{.2\textwidth}
\includegraphics[height=9ex]{rich-sig}\\[2ex]
\hspace*{.3\textwidth}
Richard FitzJohn

\end{document}
